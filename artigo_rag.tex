\documentclass[a4paper,12pt]{article}
\usepackage[T1]{fontenc}
\usepackage[utf8]{inputenc}
\usepackage{amsmath}
\usepackage{times} 
\usepackage{geometry}
\usepackage{fancyhdr} 
\usepackage{setspace} 
\usepackage{graphicx}
\usepackage{hyperref}
\usepackage{cite}
\usepackage{indentfirst}

\geometry{a4paper, margin=1in}



\pagestyle{fancy}
\fancyhf{}
\renewcommand{\headrulewidth}{0.2pt} 
\fancyhead[L]{\small Proceedings of the 40\textsuperscript{th} Brazilian Symposium on Data Bases}
\fancyhead[R]{\small April 2025 – Santa Maria, RS, Brazil}

\begin{document}

\vspace{2cm}

\begin{center}
    {\Large\bfseries Evaluation of the Efficiency of \\
    Retrieval-Augmented Generation Systems\\
    for Solving Programming Problems \\[1em]}
    
    \normalsize
    Gabriel Machado Lunardi, Gabriel Souza Baggio\textsuperscript{1}\\[0.5em]
    
    \small
    \textsuperscript{1}Computer Science Department - Universidade Federal de Santa Maria - Santa Maria, RS, Brasil
\end{center}

\vspace{1cm}

\noindent
\textbf{Abstract.} \textit{
This study presents several tests on the use of Retrieval-Augmented Generation (RAG) in solving programming problems. It also contains a select programming questions database, comprising a diverse collection of challenges from a well-known competitive programming platform. The results demonstrate that incorporating relevant code examples and algorithmic explanations improves the quality and efficiency of solutions generated by large language models (LLMs), especially in problems that require specific knowledge or have harder logical approaches. The proposed methodology not only improves the accuracy of the solutions, but also provides more detailed and instructive explanations, suggesting significant potential for educational applications.
}

\vspace{1cm}

\section{Introduction}
Embora os modelos de linguagem de grande escala (LLMs) atuais tenham demonstrado capacidades impressionantes na geração de código, eles frequentemente produzem soluções incorretas ou ineficientes para problemas algorítmicos complexos devido à ausência de conhecimento contextual específico e compreensão aprofundada de estruturas de dados.

Os sistemas de Recuperação Aumentada por Geração (RAG) trazem uma abordagem promissora para superar essas limitações, combinando a capacidade generativa das LLMs com sistemas de recuperação de informação (RI) que utilizam uma base de conhecimento acurada, específica do assunto em questão. Este artigo investiga a aplicação desses sistemas RAG na resolução de problemas de programação, especificamente focados em desafios comuns de entrevistas técnicas e de competições de programação, tendo como fim o auxílio no aprendizado de estudantes que procuram uma fonte concisa de rápido e fácil acesso que garanta respostas certas e explicações acerca dos algoritmos utilizados nas questões.

\section{Related Works}
Abordar artigos que tem relação com o assunto, citar fontes

\section{Methodology}
Metodologia

\begin{thebibliography}{99}
    \bibitem{lewis2020rag} Lewis, P., et al. (2020). Retrieval-Augmented Generation for Knowledge-Intensive NLP Tasks. NeurIPS 2020.
    
    \bibitem{chen2021codex} Chen, M., et al. (2021). Evaluating Large Language Models Trained on Code. arXiv preprint arXiv:2107.03374.
    
    \bibitem{izacard2022atlas} Izacard, G., et al. (2022). Atlas: Few-shot Learning with Retrieval Augmented Language Models. arXiv preprint arXiv:2208.03299.
    
\end{thebibliography}

\end{document}